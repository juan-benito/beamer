\pagestyle{headings}
\clearpage
\chapter*{INTRODUCCIÓN}
La medición del Índice de Rugosidad Internacional (IRI) es un método utilizado para evaluar la calidad de las carreteras y la comodidad de los usuarios al transitar por ellas. El IRI se define como la medida del movimiento vertical de un vehículo que se produce debido a las irregularidades de la superficie de la carretera. Esta medida se expresa en unidades de longitud, generalmente en metros por kilómetro (m/km)o también (mm/m).

La medición del IRI se realiza mediante el uso de equipos especiales que registran el movimiento vertical del vehículo a lo largo de la carretera. Estos equipos están equipados con acelerómetros y otros sensores que miden la aceleración y la velocidad del vehículo en tiempo real. La información obtenida se procesa posteriormente mediante un software especializado que calcula el valor del IRI.

El IRI se utiliza como un indicador de la calidad de la superficie de la carretera y su influencia en el confort y la seguridad de los usuarios. Un IRI alto indica una carretera con superficie irregular y puede provocar fatiga del conductor, aumentar el consumo de combustible, generar daños en los vehículos y disminuir la vida útil de la carretera. Por lo tanto, la medición del IRI es esencial para la planificación y mantenimiento adecuado de las carreteras.

%La \acf{AI} es una rama de la ciencia de la computación que busca desarrollar sistemas que imiten la inteligencia humana. El \ac{ML} y el \acl{DL} son subcampos importantes dentro de la \ac{AI}. Además, el \acs{NLP} es otra área de investigación relevante que se enfoca en la interacción entre humanos y máquinas a través del lenguaje.


%\Gls{latex} es muy popular en el ámbito académico.\acrfull{pdf}




