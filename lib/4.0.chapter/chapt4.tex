\pagestyle{headings}
\chapter{METODOLOGÍA}
\section{Tipo y nivel de investigación}
\subsection{Tipo de la investigación}

Se presentan diversas perspectivas con respecto a la clasificación o división de la investigación. No obstante, independientemente del criterio utilizado, según \cite[]{Zacarias2020} se espera que cumpla con el principio de parsimonia, es decir, que sea exhaustivo y excluyente.

\cite[47]{SanchezCarlessi2015} destaca la importancia de estar conscientes sobre la naturaleza y los propósitos de la investigación, clasificándolos en tres categorías. En primer lugar, se encuentra la investigación básica, también conocida como pura o fundamental, según lo menciona el autor. En este tipo de investigación, el investigador sostiene que \emph{"[...] Mantiene como propósito recoger información de la realidad para enriquecer el conocimiento científico, está orientado al descubrimiento de principios y leyes”}. En segunda instancia, nos encontramos con la investigación aplicada, también denominada constructiva o utilitaria. En este caso, la investigación aplica los conocimientos teóricos a una situación específica para obtener consecuencias prácticas derivadas de la deducción. Por último, se aborda la investigación sustantiva, donde el enfoque está dirigido a describir, explicar y prever la realidad.

Por otra parte, \cite[36]{Tamayo2004} clasifica la investigación según su enfoque. Por un lado, se encuentra la investigación pura, también denominada básica o fundamental. Esta modalidad se caracteriza por buscar confrontar la teoría con la realidad, con la finalidad de desarrollar una teoría basada en leyes y principios. Por otro lado, está la investigación aplicada, también conocida como activa o dinámica. Su particularidad radica en que se centra en la aplicación inmediata de conocimientos, sin enfocarse en el desarrollo de teorías.

Ahora bien, al considerar la revisión bibliográfica de los dos autores, se determina que la presente investigación adopta un enfoque básico, puro o fundamental. En este contexto, se llevará a cabo la recopilación de datos existentes en el Gobierno Regional de Apurímac.
\subsection{Nivel de la investigación} \label{nivel_invest}

\cite[88]{RiosRamirez2017} clasifica el nivel de investigación según los propósitos o finalidades, la naturaleza de los datos o enfoque, el nivel de conocimiento, las fuentes de información, las condiciones de información, el tiempo y el diseño o control de las variables. En este caso, es necesario profundizar en el criterio del nivel de conocimiento, según el autor, quien afirma que "[...] involucra el grado de conocimiento sobre el objeto de estudio". Al mismo identifica cuatro tipos de investigación según esta clasificación.El primero es el exploratorio, un estudio en el cual no se puede distinguir claramente entre las variables independientes o dependientes, de ahi que también se le conoce como univariado. En este tipo de investigación, se examinan temas poco abordados. El descriptivo, por su parte, "[...] busca encontrar las características, comportamiento y propiedades del objeto de estudio" en el tiempo presente o futuro. La investigación relacional hace mérito a su nombre, mide la relación entre dos o más variables dadas, pero es importante señalar que no determina la causa-efecto. Por último, la investigación explicativa sí determina la relación causa-efecto.

Por otra parte, si bien es cierto que \cite[93]{HernandezSampieri2014} coincide con \citeauthor{RiosRamirez2017} en la mayoría de las denominaciones de su clasificación, el autor utiliza la denominación "correlacional" en lugar de "relacional". Muchos autores como \cite[]{Zacarias2020} opinan que debería llamarse "relacional", ya que la relación implica una conexión entre variables, mientras que la correlación se refiere a la relación entre unidades.

\begin{figure}[ht]
\captionsetup{width=0.95\textwidth}
\centering
\includegraphics[width=0.95\textwidth]{4.3.pdf}
\caption[Niveles de investigación]{Niveles de investigación \\ Fuente: Adaptado de \citeauthor[]{Zacarias2020}}
\label{fig:niv_inv}
\end{figure}

Conocidas los diferentes puntos de vista, la presente investigación es de nivel relacional, dado que se buscara identificar las relaciones entre las variables valor ganado y las modalidades de ejecución de obras según las normativas peruanas.

\section{Diseño de la investigación}
\cite[128]{HernandezSampieri2014} explica el diseño de la investigación como el "plan o estrategia que se desarrolla para obtener la información requerida en una investigación". El autor distingue dos diseños, ambos con una importancia destacada. 

En primer lugar, está el diseño experimental, donde las variables independientes se manipulan deliberadamente para observar sus efectos sobre otras variables dependientes en una situación controlada. A su vez, este diseño se subdivide en tres clases. La característica distintiva de todos ellos es que deben tener validez interna y externa, es decir, que los resultados deben ser aplicables primero a la población en estudio y luego al entorno externo. Las tres subclases son: preexperimentos, experimentos puros y cuasiexperimentos. No profundizaremos en estos enfoques ya que no es el objetivo de este estudio.

En contraste, los diseños no experimentales son aquellos estudios en los cuales solo se observan los fenómenos, y no se realiza una manipulación deliberada de variables. Este tipo de diseño, según su dimensión temporal o puntos en los cuales se recolectan datos, se puede clasificar en investigación transeccional, también conocida como transversal, la cual se caracteriza por recopilar datos una sola vez y en un solo momento. A su vez, se clasifica en tres subcategorías: exploratorios, descriptivos y correlacionales-causales, basándose en lo mencionado en la sección \ref{nivel_invest}. Por otro lado, la investigación longitudinal o evolutiva obtiene datos en diferentes puntos en el tiempo, ya que compara los resultados a través del cambio.\cite[341]{NaupasPaitan2014} clasifica este diseño en descriptiva simple, descriptiva comparativa, causal-comparativa, correlacional, longitudinal y transversal.

En cuanto a los criterios de planificación para la toma de datos en el tiempo, no se encontraron autores que mencionen explícitamente esta clasificación. Sin embargo, se sobreentiende que si los datos fueron planificados previamente, el estudio será prospectivo. En cambio, si los datos ya estaban registrados, el estudio será retrospectivo.
\vspace{5mm}
\begin{figure}[h]
\centering
\captionsetup{width=0.95\textwidth}
\includegraphics[width=0.95\textwidth]{4.2.pdf}
\caption[Diseño de investigación]{Diseño de investigación y los tipos existentes \\ Fuente: Adaptado de \cite[155]{HernandezSampieri2014} }
\label{fig:dis_inv}
\end{figure}

Tomando en consideración la revisión bibliográfica, la presente investigación adoptará un diseño no experimental, ya que no se manipularán las variables. Será del tipo transeccional o transversal, dado que los datos se recopilarán en un único momento. En cuanto a la planificación, la recopilación de datos se llevará a cabo de manera retrospectiva, ya que estos ya existen en el Sistema de Seguimiento  de  Inversiones (\acrshort{ssi}), del \acrshort{mef} y otros documentos.

\section{Descripción ética de la investigación}

El presente trabajo de investigación en cuanto a las citaciones de referencias bibliográficas sigue las directrices de las normas establecidas por \cite[]{IOS2021}, mas conocidos como ISO-690 , en cuanto a la redacción siguen las pautas proporcionadas por el formato del Vicerrectorado de investigación de la Universidad Nacional Micaela Bastidas de Apurímac.

Por otra parte se tomó en cuenta con lo dispuesto por el \cite[]{VRI2018}, los titulos que corresponden a las normas de comportamiento de quienes investiguen en la página 2, buenas prácticas de los investigadores y la investigación con personas.

\section{Población y muestra}

Según \cite[89]{AriasGonzales2020}, la población es un conjunto, ya sea infinito o finito, de sujetos con características similares o comunes entre sí. Además, destaca la importancia de comprender, describir o generalizar sobre este grupo total mediante la investigación. La mención de trabajar con una muestra representativa también es fundamental. Dado que estudiar toda la población puede ser impráctico, la elección de una muestra adecuada es esencial para hacer inferencias válidas sobre la población en su conjunto.Otros autores lo denominan también como  universo y se representa con (N).

Por otra parte la muestra es una parte seleccionada de la población que se utiliza para realizar inferencias sobre la población en su conjunto. \cite[173]{HernandezSampieri2014} menciona que la muestra es un \emph{"subgrupo del universo o población del cual se recolectan los datos y que debe ser representativo de ésta"}, al procedimiento donde se realiza la selección de la muestra se le llama muestreo, para esto la población deberá estar bien definida y delimitada.

El muestreo consisten en unos procedimientos técnicos y son de dos tipos: El primero es muestreo probabilístico donde:

De acuerdo a \cite[]{SanchezCarlessi2015}, el muestreo probabilístico se refiere a la situación en la cual \emph{"[...] puede calcularse previamente la probabilidad de poder obtener cada una de las muestras posibles"}, o como menciona \citeauthor{HernandezSampieri2014}, este tipo de muestra puede consistir en un \emph{"subgrupo de la población en el que todos los elementos tienen la misma posibilidad de ser elegidos"}. Los autores describen diversos métodos de muestreo probabilístico; sin embargo, en este trabajo de investigación nos enfocaremos exclusivamente en aquellos explicados por \citeauthor{HernandezSampieri2014}.

Uno de estos métodos es la muestra estratificada, en la cual la población se divide en varios segmentos, y la muestra seleccionada se toma de cada segmento. Por otro lado, encontramos el muestreo probabilístico por racimos, clusters o conglomerados, donde las unidades encapsuladas se agrupan en determinados lugares físicos.

En contraste, las muestras no probabilísticas se caracterizan por desconocer o no tener información acerca de la probabilidad de selección de cada elemento de la población. A su vez, este tipo de muestras se clasifican en diversas categorías, sin embargo, no serán abordadas en el presente estudio.

Para el caso del presente estudio, la población (\(N\)) está conformada por 1015 obras ejecutadas por el Gobierno Regional de Apurímac, siendo la unidad de análisis y la unidad muestral las modalidades ejecutadas tanto por administración directa como por contratación. Cabe aclarar que se excluyen las modalidades de obras por impuestos y por núcleo ejecutor, dado que sus números no alcanzan los intervalos de selección sistemática (\(K\)).

\[
n = \dfrac{Z^{2} \cdot p \cdot q }{E^{2}}
\]

Dónde:\\
\noindent n = Tamaño de la muestra\\
p = Probabilidad que la hipótesis sea verdadera\\
q = (1-p) Probabilidad de no ocurrencia de la hipótesis\\
E = Error estimado por estudiar una muestra en lugar de toda la población aceptable\\
Z = Coeficiente de confiabilidad que corresponde a una distribución normal depende del \% de confianza

Como en este caso  $ p = q = 50\%  $, dado que no existen previos a este trabajo.

Con estas consideraciones, el tamaño de la muestra es de 26 obras ejecutadas por el Gobierno Regional de Apurímac. El muestreo se realiza mediante una muestra estratificada con criterio proporcional. De esta manera, las muestras de obras ejecutadas por administración directa son de 12, mientras que las ejecutadas por contratación son 14. \citeauthor{HernandezSampieri2014} s

\section{Procedimiento}

El procedimiento de la investigación se redactó teniendo en cuenta el orden secuencial del diagrama de flujo mostrado a continuación: 

\begin{figure}[h]
\captionsetup{width=0.95\textwidth}
\centering
\includegraphics[width=0.95\textwidth]{4.1.pdf}
\caption[Procedimiento investigativo]{Diagrama de flujo del procedimiento de investigación}
\label{fig:proced}
\end{figure}

\section{Técnicas e instrumentos}

En el presente trabajo se utilizarán fichas de recolección de datos para procesar información obtenida del \acrlong{ssi}, infoobras y el portal de transparencia del \acrlong{gore}. Cabe mencionar que muchos autores no consideran estas fichas como un instrumento de recolección de datos propiamente dicho, ya que se recopilarán datos de fuentes secundarias. Estos datos incluirán la dinámica de la triple restricción, como Adicionales de obra, mayores metrados, deductivos, ampliaciones de plazo, paralizaciones y partidas nuevas.

%También se utilizarán las encuestas  como instrumento de medición, según \cite{HernandezSampieri2014}, 
\section{Estadístico de investigación}

La presente investigación seguirá la técnica estadística denominada T-dstudent si el análisis de datos tiene un comportamiento paramétrico, es decir si cumple con la distribución normal.