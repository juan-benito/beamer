%\renewcommand{\arraystretch}{1.5}
\pagestyle{plain}
% Configurar la numeración de las tablas en el apéndice
\renewcommand{\thetable}{A\Alph{chapter}.\arabic{table}}
\setcounter{table}{0}

\appendix % Indica el comienzo de los anexos
\renewcommand{\thefigure}{A.\arabic{figure}} % Cambia el formato de la numeración de las figuras en los anexos

\begin{landscape}
    \chapter{Apéndice A: Matríz de consistencia}
    % Table generated by Excel2LaTeX from sheet 'Matriz de consistencia (2)'
    \begin{longtable}{p{6cm}p{5.5cm}p{5.5cm}p{5.5cm}}
        %\caption{Matriz de consistencia} \\
        \toprule
        \textbf{PROBLEMAS} & \textbf{OBJETIVOS} & \textbf{HIPÓTESIS} & \textbf{VARIABLES} \\
        \midrule
        \endfirsthead
        
        \multicolumn{4}{c}{{\tablename\ \thetable{}: Matriz de consistencia (continuación)}} \\
        \toprule
        \textbf{PROBLEMAS} & \textbf{OBJETIVOS} & \textbf{HIPÓTESIS} & \textbf{VARIABLES} \\
        \midrule
        \endhead
        
        \midrule
        \multicolumn{4}{r}{{Continúa en la siguiente página}} \\
        \endfoot
        
        \bottomrule
        \endlastfoot
        
        \textbf{Problema general} & \textbf{Objetivo general} & \textbf{Hipótesis general} & \textbf{Variable aleatoria} \\
        \midrule
                    
        ¿Existen diferencias en las dimensiones de la Gestión del Valor Ganado entre las distintas modalidades de ejecución de proyectos de inversión pública del Gobierno Regional de Apurímac durante el periodo 2018 al 2022? & Comparar las diferencias significativas en la Gestión del Valor Ganado entre diferentes modalidades de ejecución en proyectos de inversión pública del Gobierno Regional de Apurímac durante el periodo 2018 al 2022. & Existe una diferencia significativa en la Gestión del Valor Ganado entre diferentes modalidades de ejecución en proyectos de inversión pública del Gobierno Regional de Apurímac durante el periodo 2018 al 2022. & Gestión del Valor Ganado:  Variaciones, Índices de rendimiento y pronósticos \\
        
        \midrule
        \textbf{Problemas específicos} & \textbf{Objetivos específicos} & \textbf{Hipótesis específicas} & \textbf{Variable fija} \\
        \midrule
        %\endhead
        
        ¿Existe una diferencia en las variaciones según modalidades de ejecución por administración directa y contratación en proyectos de inversión pública del Gobierno Regional de Apurímac durante el periodo 2018 al 2022? & Comparar las diferencias significativas en las variaciones entre diferentes modalidades de ejecución por administración directa y contratación en proyectos de inversión pública del Gobierno Regional de Apurímac durante el periodo 2018 al 2022. & Existe una diferencia significativa entre las variaciones entre diferentes modalidades de ejecución por administración directa y contratación en proyectos de inversión pública del Gobierno Regional de Apurímac durante el periodo 2018 al 2022. & Modalidad de ejecución: Administración directa y por contratación \\
        
        \midrule
        ¿Existe una diferencia en los índices de rendimiento según las modalidades de ejecución por administración directa y contratación en proyectos de inversión pública del Gobierno Regional de Apurímac durante el periodo 2018 al 2022? & Comparar las diferencias significativas en los índices de rendimiento entre diferentes modalidades de ejecución por administración directa y contratación en proyectos de inversión pública del Gobierno Regional de Apurímac durante el periodo 2018 al 2022. & Existe una diferencia significativa entre los índices de rendimiento entre diferentes modalidades de ejecución por administración directa y contratación en proyectos de inversión pública del Gobierno Regional de Apurímac durante el periodo 2018 al 2022. &  \\
        
        \midrule
        ¿Existe una diferencia en los pronósticos según las modalidades de ejecución por administración directa y contratación en proyectos de inversión pública del Gobierno Regional de Apurímac durante el periodo 2018 al 2022? & Comparar las diferencias significativas en los pronósticos entre diferentes modalidades de ejecución por administración directa y contratación en proyectos de inversión pública del Gobierno Regional de Apurímac durante el periodo 2018 al 2022. & Existe una diferencia significativa entre los pronósticos entre diferentes modalidades de ejecución por administración directa y contratación en proyectos de inversión pública del Gobierno Regional de Apurímac durante el periodo 2018 al 2022. &  \\
        
        \caption{Matriz de consistencia} \\
    \end{longtable}%
\end{landscape}
	

\begin{landscape}
    \chapter{Apéndice B: Matríz de operacionalización de variables}
    \begin{longtable}{p{3cm}p{5.5cm}p{4.4cm}p{2.5cm}p{4.8cm}p{1.8cm}}
        \caption{Matríz de operacionalización} \\
        \toprule
        \textbf{Variables} & \textbf{Definición conceptual} & \textbf{Dedinición operacional} & \textbf{Dimensiones} & \textbf{Indicadores} & \textbf{Items} \\
        \midrule
        \endfirsthead
        
        \multicolumn{6}{c}%
        {{\tablename\ \thetable{} -- Continuación}} \\
        \toprule
        \textbf{Variables} & \textbf{Definición conceptual} & \textbf{Definición operacional} & \textbf{Dimensiones} & \textbf{Indicadores} & \textbf{Items} \\
        \midrule
        \endhead
        
        \midrule
        {Continúa en la siguiente página} \\
        \endfoot
        
        \bottomrule
        \endlastfoot
        
        Gestión del Valor Ganado & La Gestión del Valor Ganado implica la medición del desempeño del proyecto en términos de valor planificado (presupuesto autorizado), valor ganado (trabajo completado) y costos reales. & Se recolectarán información secundaria de las revisiones documentales de las liquidaciones de obras realizadas en el periodo 2018-2022 del Gobierno Regional de Apurímac & Variaciones & Variación del Cronograma (Schedule Variance, SV) & 1 \\
        & & & & Variación del Costo (Cost Variance, CV) & 2 \\
        & & & Indices de rendimiento & Índice de Rendimiento del Cronograma (Schedule Performance Index, SPI) & 3 \\
        & & & & Índice de Rendimiento del Costo (Cost Performance Index, CPI) & 4 \\
        & & & & Índice del Rendimiento hasta Concluir (To Complete Performance Index, TCPI) & 5 \\
        & & & Pronósticos & Estimado a la Conclusión (Estimate at Completion, EAC) & 6 y 7 \\
        & & & & Estimado hasta concluir (Estimate to Complete, ETC) & 7 y 8 \\
        & & & & Variación a la Conclusión (Variance at Completion, VAC) & 9 y 10 \\
        & & & & Índice de Rendimiento del Costo a la Conclusión (Cost Performance Index at Conclusion, CPIAC) & 11 y 12 \\ \midrule
        Modalidades de ejecución & Las modalidades de ejecución se rigen por normativas específicas que establecen los marcos legales y procedimientos a seguir.Las normas que regunlan dichos criterios son \citetitle[]{CGR1988} y \citetitle[]{CRP2018a} & No se recogeran datos ya que se tratan de variables nominales & Administracion indirecta & Ley de contrataciones del Estado N° 30225 &  \\
        & & & Administración directa & Resolución de contraloría N° 195-88-CG &  \\
    \end{longtable}
\end{landscape}